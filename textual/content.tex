% =========================
% SEÇÃO 1 – INTRODUÇÃO
% =========================

\section{Introdução}

% Slide de abertura da seção
\begin{frame}[plain]
    \vfill
    \centering
    \begin{beamercolorbox}[sep=8pt,center,shadow=true,rounded=true]{title}
        \usebeamerfont{title}\insertsectionhead\par
        \color{oxfordblue}\noindent\rule{10cm}{1pt} \\
        \LARGE{\faLightbulbO}
    \end{beamercolorbox}
    \vfill
\end{frame}

% Slide: Da prudência ao Homo Economicus
\subsection{Da prudência ao Homo Economicus}
\begin{frame}{Da prudência ao \textit{Homo Economicus}}
    \begin{itemize}
        \item \textbf{Economia “pré-formal”:} de Adam Smith à economia política clássica, o ser humano aparece com paixões, hábitos, moral e limitações.
        \item \textbf{Século XX:} a formalização neoclássica troca prudência por cálculo: agente maximizador, preferências estáveis, informação perfeita.
        \item \textbf{Modelo canônico:} equilíbrio geral, expectativas “bem-comportadas” e um mundo em que errar é exceção, não regra.
        \item Ideia implícita: se os indivíduos calculam direito, \textbf{o coletivo “herda” a racionalidade} — bastaria somar as decisões.
    \end{itemize}
\end{frame}

% Notas do apresentador:
% - Começar lembrando que a economia não nasceu com o agente-robô dos livros de macro.
% - Citar Smith (TMS) como exemplo de autor que leva a sério virtudes, paixões e limites cognitivos.
% - Contrapor com o movimento de “limpeza” conceitual da teoria da escolha racional: tirar fricções, tirar ruídos, deixar só otimização.
% - Enfatizar que isso foi útil matematicamente, mas plantou a semente da racionalidade plena como padrão da macro moderna.

% Slide: Kahneman e o retorno da (i)racionalidade
\subsection{Kahneman e o retorno da (i)racionalidade}
\begin{frame}{Kahneman e o retorno da (i)racionalidade}
    \begin{itemize}
        \item \textbf{Herbert Simon (1955):} racionalidade limitada, custos de informação, regras de bolso em vez de ótimo perfeito.
        \item \textbf{Kahneman \& Tversky:} heurísticas, vieses sistemáticos, \textit{prospect theory} — não erramos “por acaso”, erramos em padrão.
        \item A economia comportamental mostra: \textbf{o humano real} é míope, avesso a perdas, enviesado, influenciável pelo enquadramento.
        \item Paradoxo: incorporamos isso em micro, finanças, psicologia econômica… mas a macro \textbf{continua modelando o agente como se nada tivesse acontecido}.
    \end{itemize}
\end{frame}

% Notas do apresentador:
% - Situa Simon como a primeira rachadura séria no ideal de racionalidade plena.
% - Explicar em 1 minuto a ideia de heurísticas e vieses (âncora, disponibilidade, representatividade).
% - Fazer o gancho: “Se os indivíduos são assim, por que nossos modelos macro ainda fingem que eles são calculadoras perfeitas?”

% Slide: Quando os modelos param no tempo
\subsection{Quando os modelos param no tempo}
\begin{frame}{Quando os modelos param no tempo}
    \begin{itemize}
        \item A macro moderna consolida o pacote: \textbf{expectativas racionais + otimização intertemporal + equilíbrio geral dinâmico}.
        \item Nasce o paradigma DSGE como \textit{língua franca} da macro de bancos centrais e academia.
        \item A evidência comportamental avança, mas os modelos dominantes \textbf{relutam em atualizar a imagem do agente}.
        \item Fica a pergunta: \textbf{como fazer macro que leve a sério a (i)racionalidade} sem jogar fora a estrutura de modelos dinâmicos?
    \end{itemize}
\end{frame}

% Notas do apresentador:
% - Descrever rapidamente o que é um DSGE: agentes otimizadores, mercados que se equilibram, choques exógenos.
% - Lembrar que essa arquitetura foi muito útil, mas virou “doutrina oficial”.
% - Conclusão da seção: a teoria de decisão evoluiu, mas a macro-modelagem ficou defasada. É aqui que entram os modelos baseados em agentes e o artigo que vamos analisar.

% =========================
% SEÇÃO 2 – DSGE EM XEQUE
% =========================

\section{DSGE em Xeque}

\begin{frame}[plain]
    \vfill
    \centering
    \begin{beamercolorbox}[sep=8pt,center,shadow=true,rounded=true]{title}
        \usebeamerfont{title}\insertsectionhead\par
        \color{oxfordblue}\noindent\rule{10cm}{1pt} \\
        \LARGE{\faBalanceScale}
    \end{beamercolorbox}
    \vfill
\end{frame}

% Slide: DSGE em 30 segundos
\subsection{DSGE em 30 segundos}
\begin{frame}{DSGE em 30 segundos}
    \begin{itemize}
        \item \textbf{D}ynamic: horizonte intertemporal longo, decisões hoje afetam o futuro.
        \item \textbf{S}tochastic: choques exógenos (tecnologia, política monetária, preferências).
        \item \textbf{G}eneral \textbf{E}quilibrium: mercados se limpam, preços ajustam, não há desequilíbrios persistentes.
        \item Agentes: normalmente representativos, com \textbf{expectativas racionais} e racionalidade plena.
        \item Resultado: um modelo elegante, tratável mas que \textbf{empurra a complexidade e a irracionalidade para fora do quadro}.
    \end{itemize}
\end{frame}

% Notas do apresentador:
% - Explicar com linguagem simples: “é o modelo em que todo mundo sabe tudo, otimiza tudo, e o economista acompanha a dança resolvendo equações”.
% - Comentar que muitas versões modernas já tentam melhorar isso, mas o núcleo ainda é esse.

% Slide: Quatro críticas centrais aos DSGE
\subsection{Quatro críticas centrais aos DSGE}
\begin{frame}{Quatro críticas centrais aos DSGE (visão ACE)}
    \begin{itemize}
        \item \textbf{Heterogeneidade ignorada:} agente representativo apaga diferenças de renda, tecnologia, informação e rede de interações.
        \item \textbf{Desequilíbrio excluído:} foco em trajetórias de equilíbrio; crises, desemprego persistente e travamentos são difíceis de modelar.
        \item \textbf{Complexidade domesticada:} linearização em torno de um ponto “ideal” pode matar dinâmicas não-lineares e efeitos de rede.
        \item \textbf{Racionalidade plena:} expectativas racionais e otimização estrita, pouco espaço para heurísticas, aprendizado, erros sistemáticos.
    \end{itemize}
\end{frame}

% Notas do apresentador:
% - Dizer que essas são exatamente as quatro críticas organizadas pelo artigo: heterogeneidade, desequilíbrio, complexidade e racionalidade.
% - Já preparar o terreno para a próxima seção: os modelos baseados em agentes surgem tentando enfrentar essas quatro frentes ao mesmo tempo.

% =========================
% SEÇÃO 3 – MACRO-ACE
% =========================

\section{Macroeconomia Baseada em Agentes}

\begin{frame}[plain]
    \vfill
    \centering
    \begin{beamercolorbox}[sep=8pt,center,shadow=true,rounded=true]{title}
        \usebeamerfont{title}\insertsectionhead\par
        \color{oxfordblue}\noindent\rule{10cm}{1pt} \\
        \LARGE{\faUsers}
    \end{beamercolorbox}
    \vfill
\end{frame}

% Slide: O que é macro-ACE?
\subsection{O que é macro-ACE?}
\begin{frame}{O que é macro-ACE?}
    \begin{itemize}
        \item \textbf{Agent-based Computational Economics (ACE):} simulações com muitos agentes heterogêneos que interagem em rede.
        \item Cada agente segue \textbf{regras simples}: heurísticas de consumo, investimento, crédito, expectativas adaptativas.
        \item A ênfase está em \textbf{fenômenos emergentes}: ciclos, crises, bolhas, desemprego — como propriedades do sistema.
        \item Diferente do DSGE, não se assume equilíbrio geral a priori; desequilíbrios e ajustamentos fazem parte da dinâmica.
    \end{itemize}
\end{frame}

% Notas do apresentador:
% - Comparar com “jogo de simulação”: programar famílias, firmas, bancos e deixar o sistema rodar.
% - Ressaltar a ideia de emergência: o “macro” nasce das interações, não de uma equação agregada mágica.

% Slide: Quatro famílias de modelos macro-ACE
\subsection{Quatro famílias de modelos macro-ACE}
\begin{frame}{Quatro famílias de modelos macro-ACE}
    \begin{itemize}
        \item \textbf{K\&S (Keynes \& Schumpeter):} firmas, bancos e inovação endógena; ciclos de negócios emergem de demanda, crédito e tecnologia.
        \item \textbf{CATS:} foco em complexidade e redes financeiras; choques se propagam via encadeamentos produtivos e de crédito.
        \item \textbf{EURACE:} “simulador” da economia europeia, com famílias, firmas, bancos e governo em múltiplas regiões.
        \item \textbf{Strategy-Switching:} agentes alternam entre estratégias (fundamentalistas vs. chartistas), gerando bolhas e crashes endógenos.
    \end{itemize}
\end{frame}

% Notas do apresentador:
% - Não entrar em detalhes técnicos; a ideia é mostrar a diversidade de programas de pesquisa.
% - Explicar que todos respondem, cada um à sua maneira, às quatro críticas feitas aos DSGE.

% Slide: O que esses modelos explicam?
\subsection{O que esses modelos explicam?}
\begin{frame}{O que esses modelos explicam?}
    \begin{itemize}
        \item \textbf{Ciclos de negócios endógenos:} sem precisar de grandes choques exógenos para gerar flutuações macro.
        \item \textbf{Distribuições realistas:} tamanhos de firmas, produtividade, desemprego por região/setor, etc.
        \item \textbf{Instabilidade financeira:} bolhas de crédito, cascatas de falência, crises sistêmicas como emergências de rede.
        \item \textbf{Efeitos de política:} impacto de políticas fiscais e monetárias em ambientes com heterogeneidade e restrições de crédito.
    \end{itemize}
\end{frame}

% Notas do apresentador:
% - Destacar que o artigo revê justamente evidências desse tipo: calibração, comparação de momentos, reprodução de “fatos estilizados”.
% - Concluir: ACE não é só “brincadeira de computador”; é uma tentativa séria de modelar um mundo com racionalidade limitada.

% ===========================================
% SEÇÃO 4 – O ARTIGO: DILAVER, JUMP & LEVINE
% ===========================================

\section{Agent-based macroeconomics and DSGE models}

\begin{frame}[plain]
    \vfill
    \centering
    \begin{beamercolorbox}[sep=8pt,center,shadow=true,rounded=true]{title}
        \usebeamerfont{title}\insertsectionhead\par
        \color{oxfordblue}\noindent\rule{10cm}{1pt} \\
        \LARGE{\faExchange}
    \end{beamercolorbox}
    \vfill
\end{frame}

% Slide: O projeto do artigo
\subsection{O projeto do artigo}
\begin{frame}{O projeto do artigo}
    \begin{itemize}
        \item Artigo de \textbf{Dilaver, Jump \& Levine (2018)}: “Agent-based macroeconomics and DSGE models: Where do we go from here?”.
        \item \textbf{Objetivo central:} revisar e sintetizar a literatura de macro-ACE e conectá-la ao mundo DSGE.
        \item Três blocos principais:
        \begin{itemize}
            \item Críticas ACE aos DSGE.
            \item Principais modelos macro-ACE e seu desempenho empírico.
            \item Modelos New Keynesian comportamentais e o NK “internally rational”.
        \end{itemize}
        \item Pergunta de fundo: \textbf{é possível construir uma macro mais realista} combinando o melhor dos dois mundos?
    \end{itemize}
\end{frame}

% Notas do apresentador:
% - Deixar claro que é um artigo de síntese, não de modelo novo.
% - Falar que a força dele é justamente organizar um campo disperso e ligar ACE e DSGE em uma narrativa coerente.

% Slide: DSGE x ACE – o mapa mental do artigo
\subsection{DSGE x ACE – o mapa mental}
\begin{frame}{DSGE x ACE – o mapa mental do artigo}
    \begin{itemize}
        \item \textbf{Passo 1:} organizar as quatro críticas ACE aos DSGE (heterogeneidade, desequilíbrio, complexidade, racionalidade).
        \item \textbf{Passo 2:} mostrar como os modelos K\&S, CATS, EURACE e Strategy-Switching lidam com cada uma delas.
        \item \textbf{Passo 3:} avaliar o quanto esses modelos reproduzem “fatos estilizados” macroeconômicos.
        \item \textbf{Passo 4:} discutir tentativas de síntese via modelos New Keynesian comportamentais e NK internamente racionais.
    \end{itemize}
\end{frame}

% Notas do apresentador:
% - Você pode desenhar verbalmente um diagrama mental: de um lado DSGE, do outro ACE, e o artigo no meio como “ponte”.
% - Ressaltar a honestidade intelectual dos autores: não é um panfleto anti-DSGE, é uma avaliação crítica, mas construtiva.

% Slide: New Keynesian comportamentais
\subsection{New Keynesian comportamentais}
\begin{frame}{New Keynesian comportamentais}
    \begin{itemize}
        \item Ideia básica: \textbf{injetar comportamento “imperfeito”} dentro do arcabouço NK padrão.
        \item Exemplos:
        \begin{itemize}
            \item Expectativas adaptativas ou heterogêneas em vez de totalmente racionais.
            \item Regras simples de consumo e investimento (heurísticas) em vez de solução exata do problema dinâmico.
        \end{itemize}
        \item Resultado: modelos ainda resolvidos por técnicas DSGE, mas com \textbf{canais comportamentais} para volatilidade e desajustes.
        \item Interpretação do artigo: caminho promissor, porém ainda tímido na direção da complexidade e da heterogeneidade “full ACE”.
    \end{itemize}
\end{frame}

% Notas do apresentador:
% - Deixar claro que aqui o compromisso é com a “forma DSGE”, mas flexibilizando suposições de expectativas e decisão.
% - Comentar que esses modelos são mais palatáveis para bancos centrais, porque preservam boa parte da estrutura NK.

% Slide: Modelo NK internamente racional
\subsection{Modelo NK internamente racional}
\begin{frame}{Modelo NK internamente racional}
    \begin{itemize}
        \item Ponto de partida: trabalhos de Adam \& Marcet sobre \textbf{internal rationality}.
        \item Agentes maximizam \textbf{dado o modelo mental que possuem} – mas esse modelo pode ser incompleto ou errado sobre o “verdadeiro” processo.
        \item Em vez de onisciência, há \textbf{aprendizado e atualização}: expectativas são racionais “por dentro”, mas não necessariamente corretas “por fora”.
        \item Dilaver et al. veem essa linha como tentativa de dar \textbf{microfundamentos mais realistas} à macro comportamental, sem abrir mão da linguagem NK.
    \end{itemize}
\end{frame}

% Notas do apresentador:
% - Explicar com analogia: o agente é racional “dentro da bolha de conhecimento” dele, não em relação a todo o universo.
% - Sugerir que isso conversa muito com a ideia de vieses e modelos mentais da economia comportamental.

% Slide: Para onde vamos?
\subsection{Para onde vamos?}
\begin{frame}{Para onde vamos?}
    \begin{itemize}
        \item O artigo sugere que não há “modelo final”, mas \textbf{um programa de pesquisa}:
        \begin{itemize}
            \item ACE para capturar heterogeneidade, redes, desequilíbrios e dinâmica rica.
            \item NK comportamentais/IR para manter estrutura analítica e dialogar com a prática de política econômica.
        \end{itemize}
        \item Desafio técnico: \textbf{conciliar complexidade com tratabilidade} – modelos ricos o suficiente, mas ainda estimáveis e comunicáveis.
        \item Desafio conceitual: \textbf{aceitar a racionalidade limitada} de indivíduos e instituições, sem abandonar a ambição de prever e orientar políticas.
    \end{itemize}
\end{frame}

% Notas do apresentador:
% - Fechar essa seção destacando que o artigo não “mata” os DSGE, mas exige que eles convivam com modelos mais realistas.
% - Ligar com a narrativa da racionalidade: saímos da ingenuidade, passamos pelo excesso de confiança, e agora tentamos um meio-termo humilde e empírico.

% =========================
% SEÇÃO 5 – CONCLUSÃO
% =========================

\section{Conclusão}

\begin{frame}[plain]
    \vfill
    \centering
    \begin{beamercolorbox}[sep=8pt,center,shadow=true,rounded=true]{title}
        \usebeamerfont{title}\insertsectionhead\par
        \color{oxfordblue}\noindent\rule{10cm}{1pt} \\
        \LARGE{\faQuestionCircle}
    \end{beamercolorbox}
    \vfill
\end{frame}

% Slide: Racionalidade, modelos e mundo real
\subsection{Racionalidade, modelos e mundo real}
\begin{frame}{Racionalidade, modelos e mundo real}
    \begin{itemize}
        \item A história que contamos:
        \begin{itemize}
            \item Começamos admitindo que o ser humano \textbf{não é plenamente racional}.
            \item Construímos modelos como se ele fosse.
            \item A psicologia e a economia comportamental lembraram que os limites voltaram à cena.
        \end{itemize}
        \item O artigo de Dilaver, Jump \& Levine mostra a macro \textbf{saindo do conforto do DSGE puro} e experimentando modelos com agentes reais.
        \item Em última análise, a pergunta não é só técnica: \textbf{que imagem de racionalidade queremos supor quando descrevemos sociedades inteiras?}
    \end{itemize}
\end{frame}

% Notas do apresentador:
% - Retomar em 1 minuto a linha do tempo: pré-racionalidade plena → homo economicus → Kahneman → crise dos DSGE → ACE e NK comportamentais.
% - Fechar com um convite: mais do que uma disputa de modelos, é uma discussão sobre como enxergamos o ser humano na macroeconomia.
